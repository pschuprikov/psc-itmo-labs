\documentclass[12pt]{article}
\usepackage[utf8]{inputenc}
\usepackage[russian]{babel}

\author{Чуприков Павел Сегреевич}
\title{Курсовая работа по теме: "Вычисление дискретной временной диаграммы Вороного с помощью графического процессора"}
\begin{document}
\begin{titlepage}
\maketitle
\end{titlepage}

\tableofcontents

\pagebreak

\section{Введение}
\emph{Диаграммы Вороного} --- широко известный и глубоко изученный предмет вычислительной геометрии. Его можно кратко описать как такой способ разбиения плоскости, где сначала выделяется некоторое множество точек-источников, и затем каждая точка плоскости "красится" в цвет ближайшей к ней точки-источника. Таким диаграммам находится множество применений в таких задачах, как определение столкновений, планирование пути, кластеризация данных, обзор применений 
можно найти в \cite{survey}. Нашли применения даже визуальные свойства такого разбиения плоскости: например, в области процедурной генерации текстур \cite{proced}. 

В некотором роде расширением диаграмм Вороного является \emph{преобразование расстояний} (поле расстояний, карта расстояний). Они отличаются тем, что последняя каждой точке ставит в соответствие не только цвет точки-источника, но и расстояние до нее (на самом деле в некоторых применениях важно только лишь расстояние). Область применимости поля расстояний включает: определение столкновений, навигацию искусственного интеллекта, отрисовка векторных данных. 

И первая и вторая геометрическая конструкции легко обобщаются на случай, когда в качестве источников выступают не точки, а отрезки, окружности или любые другие геометрические примитивы. 

Нужно отметить, что и диаграмма Вороного и поле расстояний --- это математические объекты, для практического их применения в программных продуктов, можно выделить два принципиально различных методов хранения этих объектов (или их приближений) --- это \emph{векторный} и \emph{дискретный}. Лучшая аналогия будет с векторным и растровым способами представления изображений в памяти компьютера соответственно. В первом случае поддерживается некоторая структура данных, например \emph{двусвязный список ребер} (англ. Double Connected Edge List --- DCEL), во втором случае --- область, для которой строится диаграмма Вороного, дискретизируется и для каждой единицы дискретизации цвет или расстояние хранятся отдельно.

Несмотря на то, что существует множество алгоритмов разной степени эффективности для построения как диаграмм Вороного, так и поля расстояний, в них обычно используется стандартная Евклидова метрика. Однако, часто одного расстояния недостаточно, может потребоваться фактическое время, которое необходимо затратить для перемещения из точки-источника в какую-либо точку плоскости. Заметим, что стандартная диаграмма Вороного с Евклидовой метрикой, в случае неоднородности области, не подходит . Простейший пример неоднородности можно найти при планировании маршрута: наличие дорог, скорость вдоль которых обыкновенно значительно выше, чем вне них. В данной работе описывается метод, который при некоторых (весьма строгих) упрощениях позволяет решить эту проблему для дискретного представления полей расстояний или диаграмм Вороного.

\pagebreak

\begin{thebibliography}{10}
\bibitem{survey} Franz Aurenhammer. \textit{Voronoi Diagramm --- A Survey of a Fundamental Geomteric Data Structure}.
\bibitem{proced} David S. Elbert et al. \textit{Texturing and Modelling, Third Edition: A Procedural Approach}.
\end{thebibliography}

\end{document}